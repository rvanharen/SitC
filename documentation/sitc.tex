\documentclass{memoir}

\addtolength{\textwidth}{1.3in}
\addtolength{\textheight}{2in}
\addtolength{\oddsidemargin}{-0.65in}
\addtolength{\evensidemargin}{-0.65in}
\addtolength{\topmargin}{-1in}

\usepackage{fourier} % or what ever
\usepackage[scaled=.92]{helvet}%. Sans serif - Helvetica
\usepackage{color,calc}
\usepackage{verbatim}
\newsavebox{\ChpNumBox}
\definecolor{ChapBlue}{rgb}{0.00,0.65,0.65}
\makeatletter
\newcommand
*
{\thickhrulefill}{%
\leavevmode\leaders\hrule height 1\p@ \hfill \kern \z@}
\newcommand
*
\BuildChpNum[2]{%
\begin{tabular}[t]{@{}c@{}}
\makebox[0pt][c]{#1\strut}  \\[.5ex]
\colorbox{cyan}{%
\rule[-10em]{0pt}{0pt}%
\rule{1ex}{0pt}\color{black}#2\strut
\rule{1ex}{0pt}}%
\end{tabular}}
\makechapterstyle{BlueBox}{%
\renewcommand{\chapnamefont}{\large\scshape}
\renewcommand{\chapnumfont}{\Huge\bfseries}
\renewcommand{\chaptitlefont}{\raggedright\Huge\bfseries}
\setlength{\beforechapskip}{20pt}
\setlength{\midchapskip}{26pt}
\setlength{\afterchapskip}{40pt}
\renewcommand{\printchaptername}{}
\renewcommand{\chapternamenum}{}
\renewcommand{\printchapternum}{%
\sbox{\ChpNumBox}{%
\BuildChpNum{\chapnamefont\@chapapp}%
{\chapnumfont\thechapter}}}
\renewcommand{\printchapternonum}{%
\sbox{\ChpNumBox}{%
\BuildChpNum{\chapnamefont\vphantom{\@chapapp}}%
{\chapnumfont\hphantom{\thechapter}}}}
\renewcommand{\afterchapternum}{}
\renewcommand{\printchaptertitle}[1]{%
\usebox{\ChpNumBox}\hfill
\parbox[t]{\hsize-\wd\ChpNumBox-1em}{%
\vspace{\midchapskip}%
\color{cyan}\thickhrulefill\par
\color{black}\chaptitlefont ##1\par}}%
}
\chapterstyle{BlueBox}

\flushbottom \raggedbottom

\makeatletter
% this is a copy of the original \fs@ruled but with {\color{cyan} ... } added:
\newcommand\fs@colorruled{\def\@fs@cfont{\bfseries}\let\@fs@capt\floatc@ruled
  \def\@fs@pre{{\color{cyan}\hrule height1.25pt depth0pt }\kern2pt}%
  \def\@fs@post{\kern2pt{\color{cyan}\hrule height 1.25pt}\relax}%
  \def\@fs@mid{\kern2pt{\color{cyan}\hrule height 1.25pt}\kern2pt}%
  \let\@fs@iftopcapt\iftrue}
\makeatother

\usepackage{float}

\floatstyle{colorruled}
\newfloat{program}{thp}{lop}
\floatname{program}{Program}


\begin{document}
 \chapter{Download data from \textsl{Weather Underground}}
 \section{Extract Dutch stations}
 The script \textsl{wunderground\_dump\_stationid.py} parses the Weather Underground website to find all stations in the Netherlands. For each station found, the following variables are extracted: \textsl{station id}, \textsl{neighborhood}, \textsl{city}, \textsl{station type}, and \textsl{location} (latitude, longitude, height). The zipcode of the station location is extracted using the api of Google Maps. The extracted data is saved to a csv file.
 
 The script takes one (optional) argument, which is the name of the output csv file. The default name of the output csv file is \textsl{wunderground\_stations.csv} in the current working directory.
 \begin{program}
\begin{verbatim}
usage: wunderground_dump_stationid.py [-h] [-o OUTPUT]

Extract all Wunderground stations in the Netherlands and write the station
names and locations to a csv file

optional arguments:
  -h, --help            show this help message and exit
  -o OUTPUT, --output OUTPUT
                        CSV output file [default: wunderground_stations.csv]
\end{verbatim}
\caption{wunderground\_dump\_stationid.py}
\end{program}

\section{Download Wunderground data}
Downloading data from Weather Underground is performed using the script \textsl{wunderground\_getdata.py}. A station id can be supplied to the script, using the (-s, --stationid) switch, to download data for a single station. Alternatively, the csv file that results from the \textsl{wunderground\_dump\_stationid.py} script can be supplied to the script using the (-c, --csvfile) switch to download data for all stations in the csv file.

Other relevant arguments to the script are:
\begin{itemize}
\item (-b, --startyear) and (-e, --endyear): Defines the begin and end of the period that data is downloaded for.
\item (-k, --keep): Use the argument switch to keep already downloaded data. If the argument switch is not used existing data is overwritten.
\end{itemize}
\begin{program}
\begin{verbatim}
usage: wunderground_getdata.py [-h] [-o OUTPUTDIR] [-b STARTYEAR] [-e ENDYEAR]
                               [-s STATIONID] [-c CSVFILE] [-k]
                               [-l {debug,info,warning,critical,error}]

Combine csv files weather underground in one output file

optional arguments:
  -h, --help            show this help message and exit
  -o OUTPUTDIR, --outputdir OUTPUTDIR
                        Data output directory
  -b STARTYEAR, --startyear STARTYEAR
                        Start year
  -e ENDYEAR, --endyear ENDYEAR
                        End year
  -s STATIONID, --stationid STATIONID
                        Station id
  -c CSVFILE, --csvfile CSVFILE
                        CSV data file
  -k, --keep            Keep downloaded files
  -l {debug,info,warning,critical,error}, --log {debug,info,warning,critical,error}
                        Log level
\end{verbatim}
\caption{wunderground\_getdata.py}
\end{program}

\end{document}
